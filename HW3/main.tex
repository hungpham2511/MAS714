\documentclass[a4paper,10pt,twoside]{article}
% \pdfoutput=1 

\usepackage[bookmarks=true,colorlinks=true]{hyperref}

\usepackage{bookmark}
\bookmarksetup{
 numbered, 
 open,
}


\usepackage{todonotes}

\usepackage{amssymb,amsmath}       % Equations
\usepackage{array}
\usepackage{bm}                    % Bold math symbols
\usepackage{epsfig}
\usepackage{float}
\usepackage{graphicx,color,psfrag} % Graphics, Figures
\usepackage[export]{adjustbox}
\usepackage{subcaption}
\usepackage{tikz}

\usepackage{multirow}              % For Tables
\usepackage{tabularx}              % Tables
\usepackage{wrapfig}
\usepackage[algoruled]{algorithm2e}
\SetStartEndCondition{ }{}{}%
\SetKwProg{Fn}{def}{\string:}{}
\SetKwFunction{Range}{range}%%
\SetKw{KwTo}{in}\SetKwFor{For}{for}{\string:}{}%
\SetKwIF{If}{ElseIf}{Else}{if}{:}{elif}{else:}{}%
\SetKwFor{While}{while}{:}{}%
% \renewcommand{\forcond}{$i$ \KwTo\Range{$n$}}
\AlgoDontDisplayBlockMarkers\SetAlgoNoEnd\SetAlgoNoLine%


%% enumitem 
% \labelindent is defined in both IEEEtrans and
% enumitem. \let\labelindent\relax kind-of disables \labelindent
% defined in IEEEtrans, hence avoiding the name clash.
\let\labelindent\relax
\usepackage[inline]{enumitem}

% variables
\DeclareMathOperator*{\argmin}{arg\,min}
\DeclareMathOperator*{\argmax}{arg\,max}
% acronyms
\newcommand{\ie}{{\textit{i.e.}}}
\newcommand{\etal}{\textit{et~al.}}

% theorem environment
\usepackage{amsthm}
\theoremstyle{plain}
\newtheorem{theorem}{Theorem}[section]
\newtheorem{lemma}[theorem]{Lemma}
\newtheorem{proposition}[theorem]{Proposition}
\newtheorem{corollary}[theorem]{Corollary}
\theoremstyle{definition}
\newtheorem{definition}[theorem]{Definition}
\theoremstyle{remark}
\newtheorem{remark}[theorem]{Remark}
% \newtheorem{exercise}{Exercise}[section]
\newtheorem{example}[theorem]{Example}

% Redefine exercise formating 
% \theoremstyle{plain}
\newtheorem{exercise}{Exercise}


% figures path
\graphicspath{{figures/}}

% \overrideIEEEmargins
% set margins
\setlength{\floatsep}{2pt plus 1pt minus 1pt}
\setlength{\textfloatsep}{5pt plus 1pt minus 2pt}

%% TITLE
\title{Homework 3}
%% AUTHOR
\author{Pham Tien Hung, Zhang Xu}

%% DATE
\date{}

%%%%%%%%%%%%%%%%%%%%%%%%%%%%%%%%%%%%%%%%%%%%%%%%%%%%%%%%%%%%%%%%%%%%%%
\begin{document}
\maketitle
\begin{exercise}[Sort $k$ sorted lists.]\

Idea is to use a Priority Queue (min heap).
\begin{algorithm}[h]
	\caption{Sort $k$ sorted lists ($L[1...k][:]$)}
	$PQ$ = Make-queue()\;
	Let $X$ be an empty list\;
	\For{$i$ in $[1,...,k]$}{
		Add($i$, $L[i][1]$)\;
	}
	\For{$i$ in $[1,...,n]$}{
		$j$ = Extract-min($PQ$) \tcp*{the $j$-th list}
		$x$ = pop($L[j][1]$) \tcp*{Remove the first element}
		Add $x$ to $X$\;
		Add($j$, $L[j][1]$)\;
	}
	\Return $X$\;
\end{algorithm}
\todo{Analyze Time Complexity}
\end{exercise}

\begin{exercise}[Computation of the Fibonacci number]
We shall make use of the following relation
\begin{equation}
	\begin{pmatrix}
		1 & 1\\ 1 & 0
	\end{pmatrix}
	\begin{pmatrix}F(n-1)\\F(n-2)\end{pmatrix}
	= \begin{pmatrix}F(n)\\F(n-1)\end{pmatrix}
\end{equation}
Clearly, from this we obtain
\begin{equation}
	A^n [F(0), F(1)]^T = [F(n), F(n-1)]^T
\end{equation}
where $A = \begin{pmatrix}1 & 1\\ 1 & 0\end{pmatrix}$.

Computing $A^n$ takes $O(\log n)$ matrix multiplication (of constant
size), thus the time complexity is $O(\log n)$.
\end{exercise}
\begin{exercise}[Computing maximum sum of contiguous subarray]\

\textbf{a)} Denoting $S(i, j)$ as the sum of the array starting
from the $i$-th element to the $j$-th element, that is
\[
	S(i, j) = \begin{cases}
		0 \text{ if $i < j$}\\
		\sum_{k=i}^j L[k] \text{ otherwise}\\
	\end{cases}
\]
\begin{algorithm}[H]
	\caption{Naive summation}
	Let $(1, 1)$ be the selected elements\;
	Let sum = 0\;
	Initialize S as a n-by-n zero matrix\;
	\For{$i$ in $[1,...,n]$}{
		\For{$j$ in $[1,...,n]$}{
			\If{j $>$ i}{
			S[i, j] = S[i, j - 1] + L[j]\;
			select i, j if S[i, j] is larger than current sum;
			}
		}
	}
	\Return sum
\end{algorithm}
\textbf{b)} I don't know if this is considered DP. Defining $S_l[i]$
as the sum of all element on the left of $j$ including $j$
\[
	S_l[i] = \sum_{j=1}^i L[j].
\]
\end{exercise}

Next, we find two elements $e_{min}$ and $e_{max}$ such that
\[
\begin{aligned}
	e_{min} = \argmin_i L[i],\\
	e_{max} = \argmax_i L[i].\\
\end{aligned}
\]
Now, the maximum continguous subarray is given by $[e_{min}+1,...,e_{max}]$.


Time complexities of all operations are $\Theta(n)$.
\todo{Write down the algorithm}

\begin{exercise}[Arbitrage]\

Compute the logarithm of the exhcange matrix $R[,]$, then construct
a graph based on this matrix.

Find negative cycle on this graph using Floyd-Warshall. Complexity $O(n^3)$.
	
\end{exercise}	
\end{document}























