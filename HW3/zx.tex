\documentclass[a4paper,10pt,twoside]{article}
% \pdfoutput=1 

\usepackage[bookmarks=true,colorlinks=true]{hyperref}

\usepackage{bookmark}
\bookmarksetup{
 numbered, 
 open,
}
\usepackage{todonotes}

\usepackage{amssymb,amsmath}       % Equations
\usepackage{array}
\usepackage{bm}                    % Bold math symbols
\usepackage{epsfig}
\usepackage{float}
\usepackage{graphicx,color,psfrag} % Graphics, Figures
\usepackage[export]{adjustbox}
\usepackage{subcaption}
\usepackage{tikz}

\usepackage{multirow}              % For Tables
\usepackage{tabularx}              % Tables
\usepackage{wrapfig}
\usepackage[algoruled]{algorithm2e}
\SetStartEndCondition{ }{}{}%
\SetKwProg{Fn}{def}{\string:}{}
\SetKwFunction{Range}{range}%%
\SetKw{KwTo}{in}\SetKwFor{For}{for}{\string:}{}%
\SetKwIF{If}{ElseIf}{Else}{if}{:}{elif}{else:}{}%
\SetKwFor{While}{while}{:}{}%
% \renewcommand{\forcond}{$i$ \KwTo\Range{$n$}}
\AlgoDontDisplayBlockMarkers\SetAlgoNoEnd\SetAlgoNoLine%


%% enumitem 
% \labelindent is defined in both IEEEtrans and
% enumitem. \let\labelindent\relax kind-of disables \labelindent
% defined in IEEEtrans, hence avoiding the name clash.
\let\labelindent\relax
\usepackage[inline]{enumitem}

% variables
\DeclareMathOperator*{\argmin}{arg\,min}
\DeclareMathOperator*{\argmax}{arg\,max}
% acronyms
\newcommand{\ie}{{\textit{i.e.}}}
\newcommand{\etal}{\textit{et~al.}}

% theorem environment
\usepackage{amsthm}
\theoremstyle{plain}
\newtheorem{theorem}{Theorem}[section]
\newtheorem{lemma}[theorem]{Lemma}
\newtheorem{proposition}[theorem]{Proposition}
\newtheorem{corollary}[theorem]{Corollary}
\theoremstyle{definition}
\newtheorem{definition}[theorem]{Definition}
\theoremstyle{remark}
\newtheorem{remark}[theorem]{Remark}
% \newtheorem{exercise}{Exercise}[section]
\newtheorem{example}[theorem]{Example}

% Redefine exercise formating 
% \theoremstyle{plain}
\newtheorem{exercise}{Exercise}


% figures path
\graphicspath{{figures/}}

% \overrideIEEEmargins
% set margins
\setlength{\floatsep}{2pt plus 1pt minus 1pt}
\setlength{\textfloatsep}{5pt plus 1pt minus 2pt}

%% TITLE
\title{Homework 3}
%% AUTHOR
\author{Pham Tien Hung, Zhang Xu}

%% DATE
\date{}

%%%%%%%%%%%%%%%%%%%%%%%%%%%%%%%%%%%%%%%%%%%%%%%%%%%%%%%%%%%%%%%%%%%%%%
\begin{document}
\maketitle
\begin{exercise}

Use binary heap with priority queue for implementation.

\begin{algorithm}[h]
	\caption{Merge $k$ sorted lists ($L[1][:], ..., L[k][:]$)}
	Let $BH$ be an empty heap\;
	$result = [  ]$\tcp*{output list}
	\For{$i in [1, ..., k]$}{
		Add($i$, $L[i][1]$)\tcp*{Initial heap}
	}
	\While {$BH$ is not $empty$ \do}{
		$j$ = Extract-min($BH$) \tcp*{$j$-th sorted list}
		$x$ = pop($L[j][1]$) \;
		Add $x$ to $result$\;
		if $L[j][:]$ is not $empty${\tcp*{insert only if not end of the list}
			Add($j$, $L[j][1]$)\;
		}
	}
	\Return $result$\;
\end{algorithm}

\begin{proof}
Correctness: 

The correctness follows that very time the minimum among all uninserted numbers is added to the output list of $result$.

Time Complexity:

It takes $O(k)$ to build the initial heap; for every element, it takes $O(lg k)$ for Extract-min and $O(lg k)$ to insert the next one from the same list. In total it takes $O(k + n lg k) = O(n lg k)$.

\end{proof}
\end{exercise}

\begin{exercise}
Make use of matrix multiplication.
\begin{algorithm}[h]
	\caption{$Fib(n)$}
	\If {$n = 0$}
		{\Return $0$}
	\ElseIf{$n = 1$}
		{\Return $1$}
	\Else{
		$F(0) = 0$\;
		$F(1) = 1$\;
		$A = \begin{bmatrix}1 & 1\\ 1 & 0\end{bmatrix}$\;
	}
		{\Return $A^n [F(1), F(0)]^T[0]$}\tcp*{return first element, $F(n)$}
\end{algorithm}

\begin{proof}
Correctness: 

With the matrix multiplication hint provided and definition of Fibonacci number, we can have the following relationship
\begin{equation}
	\begin{bmatrix}1 & 1\\ 1 & 0\end{bmatrix}\cdot
	\begin{bmatrix}F(n-1)\\F(n-2)\end{bmatrix}
	= \begin{bmatrix}F(n-1) + F(n-2)\\F(n-1)\end{bmatrix}
	= \begin{bmatrix}F(n)\\F(n-1)\end{bmatrix}
\end{equation}
Let $A$  denote $\begin{bmatrix}1 & 1\\ 1 & 0\end{bmatrix}$, then it's not difficult to obtain 
 \begin{equation}
	A^n \cdot \begin{bmatrix}F(1)\\F(0)\end{bmatrix} = \begin{bmatrix}F(n)\\F(n-1)\end{bmatrix}
\end{equation}

Time Complexity:

$O(n^{2.3728639}\log n)$, which is the time compexity for computing matrix multiplication.
\end{proof}
\end{exercise}


\end{document}
















