\documentclass[a4paper,10pt,twoside]{article}
\pdfoutput=1 

\usepackage{amssymb,amsmath}       % Equations
\usepackage{array}
\usepackage{bm}                    % Bold math symbols
\usepackage{epsfig}
\usepackage{float}
\usepackage{graphicx,color,psfrag} % Graphics, Figures
\usepackage{multirow}              % For Tables
\usepackage{tabularx}              % Tables
\usepackage{wrapfig}
\usepackage[ruled, linesnumbered]{algorithm2e}
\setlength{\algomargin}{2.5em}
\let\oldnl\nl% Store \nl in \oldnl
\SetStartEndCondition{ }{}{}%
\SetKwIF{If}{ElseIf}{Else}{if}{ then}{elif}{else}{}%
\SetKwFor{For}{for}{ do}{}%
\SetKwFor{ForEach}{foreach}{ do}{}%
\SetKwInOut{Input}{Input}%
\SetKwInOut{Output}{Output}%
\AlgoDontDisplayBlockMarkers%
\SetAlgoNoEnd%
\SetAlgoNoLine%
\DontPrintSemicolon

%% enumitem 
% \labelindent is defined in both IEEEtrans and
% enumitem. \let\labelindent\relax kind-of disables \labelindent
% defined in IEEEtrans, hence avoiding the name clash.
\let\labelindent\relax
\usepackage[inline]{enumitem}

%% algorithm2e
\usepackage[ruled,linesnumbered]{algorithm2e}
% remove line number for one line
\let\oldnl\nl
\newcommand{\nonl}{\renewcommand{\nl}{\let\nl\oldnl}}

%% subfigure
\usepackage[caption=false,font=footnotesize]{subfig}
% make references to subfigures appear as \thefigure(\thesubfigure)
\captionsetup[subfigure]{subrefformat=simple,labelformat=simple,listofformat=subsimple}
\renewcommand\thesubfigure{(\alph{subfigure})}

% variables
\newcommand{\mc}[2][]{{\mathcal{#2}_{\textrm{#1}}}}
\newcommand{\q}[1]{\bm{q}_{\textrm{#1}}}
\newcommand{\qd}[1]{\bm{\dot{q}}_{\textrm{#1}}}
\newcommand{\T}[1]{\bm{T}_{\textrm{#1}}}
\newcommand{\x}[1]{\bm{x}_{\textrm{#1}}}
\newcommand{\qvect}{\bm{q}}
\newcommand{\Tvect}{\bm{T}}
\newcommand{\GP}{\mathcal{G}\cap\mathcal{P}}
\newcommand{\cP}{\mathcal{P}}
\newcommand{\cC}{\mathcal{C}}
\newcommand{\cO}{\mathcal{O}}
\newcommand{\bfp}{\mathbf{p}}
% acronyms
\newcommand{\ie}{{\textit{i.e.}}}
\newcommand{\etal}{\textit{et~al.}}

% theorem environment
\newtheorem{theorem}{Theorem}
\newtheorem{proof}{Proof}
\newtheorem{lemma}{Lemma}
\newtheorem{proposition}{Proposition}
\newtheorem{corollary}{Corollary}
\newtheorem{remark}{Remark}
% TODO
\newcommand{\TODO}[1]{\noindent {\color{red} \{{\bf To-do:} #1\}}}
% COMMENT
\newcommand{\comment}[1]{}

% change tt font
\renewcommand{\tt}{\fontfamily{cmtt}\selectfont}

% figures path
\graphicspath{{figures/}}

% \overrideIEEEmargins
% set margins
\setlength{\floatsep}{2pt plus 1pt minus 1pt}
\setlength{\textfloatsep}{5pt plus 1pt minus 2pt}

%% TITLE
\title{MAS714-Homework 1}
%% AUTHOR
\author{Pham Tien Hung, Zhang Xu
}

%% DATE
\date{}


%%%%%%%%%%%%%%%%%%%%%%%%%%%%%%%%%%%%%%%%%%%%%%%%%%%%%%%%%%%%%%%%%%%%%%
\begin{document}
\maketitle
% \thispagestyle{empty}
% \pagestyle{empty}

\subsection*{Excercise 1}

The sequence is:

\begin{enumerate}
	\item f3
	\item f2
	\item f7
	\item f5
	\item f1
	\item f4
	\item f6
\end{enumerate}

\subsection*{Excercise 2}
\subsubsection*{a)} 

Answer: \emph{True}. Given $f(n)=O(g(n))$, we will prove that $\log_2f(n) = O(\log_2f(n))$.
\textbf{Proof}

First, it is noted that we assume $f(n), g(n) \geq 1$ for all $n > 0$. This
assumption ensures that the log-versions of $f$ and $g$ are non-negative,
which is a condition for the usage of the big $O$ notation.

We need to show that there exist $c > 0, n_0 \geq 0$ such that for all $n \geq n_0$,
we have
\begin{equation}
	\label{eq:f=O(g)}
	\log_2f(n) \leq c \log_2g(n).
\end{equation}
Since $f(n)=O(g(n))$, there exist $c' > 0, n_1 \geq 0$ such that for all $n \geq n_1$,
we have
\[
	f(n) \leq c' g(n).
\]
Clearly we can select $c' > 1$. Applying $\log_2$ to both sides, we have,
\[
	\log_2f(n) \leq \log_2 (c' g(n)) = \log_2 c' + \log_2 g(n).
\]
Since $g(n)$ is increasing, for $n \geq n_1$,
\[
	\log_2 g(n_1) \leq \log_2 g(n),
\]
\[
	\log_2 c' \leq \frac{\log_2 c'}{\log_2 g(n_1)}\log_2 g(n)
\]
from which implies
\[
	\log_2f(n) \leq (1 + \frac{\log_2 c'}{\log_2 g(n_1)})\log_2 g(n).
\]
We have $\log_2f(n) = O(\log_2f(n))$ follows.

\subsubsection*{b)}
Answer: \emph{False}. Here is one counterexample

Let $f(n) = 2n, g(n) = n$, clearly $2n=O(n)$. However 
\[
 	2 ^ {2n} = 4^n \notin O(2^n)
 \] 
\subsubsection*{c)}
Answer: \emph{True}.
\begin{proof}
Since $f(n)=O(g(n))$, there exist $c > 0, n_0 \geq 0$ such that for all $n \geq n_0$,
we have
\[
	f(n) \leq c g(n),
\]
which is equivalent to
\[
	f(n)^2 \leq c^2 g(n) ^2.
\]
This clearly implies $f(n)^2=O(g(n)^2)$.
\end{proof}
\subsection*{Excercise 3}
a) Counting the number of addition, we have
\[
	f(n) 
	= \sum_{i=1}^{N-1} \sum_{j=1}^i j = \sum_{i=1}^{N-1} \frac{i^2 + i}{2} 
	= \frac{1}{2} \left[ \frac{(N-1)N(2N-1)}{6}	+ \frac{N(N-1)}{2}\right]
\]
which gives us
\[
	f(n) = \frac{N^3}{3} - \frac{N}{3} = \Theta(N^3)
\]

b) 
\begin{algorithm}[h]
  \caption{New algorithm}
  \SetAlgoNoLine
  \Indm
  \Input{$A[:]$}
  \Output{$B[:, :]$}
  \Indp 
  \For{$i\in[1,\dots,n-1]$}{
	  \For{$j\in[i+1,\dots,n]$}{
	  	\If{$j==i+1$}{
	  		$B[i, j] = A[i] + A[j]$
	  	}\Else{
	  		$B[i, j] = B[i, j-1] + A[j]$
	  	}
	  }	
  }
\end{algorithm}

This algorithm has in total
\[
	g(n) = \sum_{i=1}^{N-1}i = \frac{N(N-1)}{2}=\Theta(N^2)
\]
\subsection*{Excercise 4}
\textbf{Proof}
	
	Consider a tree $T(V, E)$,
	let $V_b$ be the set of binary nodes and $V_l$ be the set of leaf node. 
	We will prove by induction on the number of nodes $|V|$.

	For $|V| = 2$, there can be no binary node while the number of leaf node
	is always 1. Therefore, this is true for this case.

	Assume that for all graphs $|V| = n$, this property holds. We now prove that for
	any graph with $|V| = (n+1)$, this property also holds. 

	Consider a tree with $n+1$ nodes, we select a leaf node in $V_l$. Its direct
	parent can be either 1) a binary node or 2) not a binary node. We now remove
	this leaf node from the original tree $T(V, E)$ to create a new tree $T'(V', E')$.
	Since $T'$ has n nodes, we know that $|V'_b| = |V'_l| - 1$.

	In case 1), we have $|V'_b| = |V_b| - 1$ and $|V'_l| = |V_l|$ -1 which implies
	\[
		|V_b| = |V_l| -1.
	\]

	In case 2), the number of binary nodes and leaf nodes remain unchanged.
	Thus, the property holds for any tree with $n+1$ nodes. With this our induction
	finishes.

\subsection*{Excercise 5}

\textbf{Proof}

	We will show that 
	\begin{enumerate}
		\item There is no cross-edges and forward-edges in a DFS 
	tree of an undirected graph.
		\item DFS(G, v) = BFS(G, v) = T implies there is no
	back-edges in the graph returned from DFS.
	\end{enumerate}
	  These are sufficient to deduce G=T.

We will now prove 1a): there is no forward-edges in a DFS tree of 
an undirected graph. 
Assume the contradiction, there exist a unlabeled edge $(u, v)$ at 
sometime $t$,
\[
	pre(u) < pre(v) < t.
\]
Now, since $t$ is when we are have already finish the recursions of
the childs of $u$ from which one is ancestor of (or is) $v$. We have
\[
 	pre(u) < pre(v) < post(v) < t
 \] 

It is clear that for a given vertex
$v$, at a time $t$ larger than $post(v)$, all undirected edges
connecting to $v$ are all labeled. This gives a contradiction since
$(u, v)$ is unlabeled at time $t > post(v)$.

The non-existence of cross-edges (1b) can be proved similarly.


1a. We will prove by contradition. Assume that there is a back-edge
in the graph G. This implies there is an unlabeled edge $(u, v)$ 
such as
\[
	pre(v) < pre(u)
\]
 
Now since T is also a BFS tree, this implies that when $v$ is visited,
$u$ is already visited or otherwise the edge $(u, v)$ would be in the tree
and not a forward-edge. This leads to a contradiction because $v$ is an
ancestor of $u$. 







\end{document}
