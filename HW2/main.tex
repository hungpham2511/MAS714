\documentclass[a4paper,10pt,twoside]{article}
\pdfoutput=1 

\usepackage{amssymb,amsmath}       % Equations
\usepackage{array}
\usepackage{bm}                    % Bold math symbols
\usepackage{epsfig}
\usepackage{float}
\usepackage{graphicx,color,psfrag} % Graphics, Figures
\usepackage{multirow}              % For Tables
\usepackage{tabularx}              % Tables
\usepackage{wrapfig}
\usepackage[algoruled]{algorithm2e}
\SetStartEndCondition{ }{}{}%
\SetKwProg{Fn}{def}{\string:}{}
\SetKwFunction{Range}{range}%%
\SetKw{KwTo}{in}\SetKwFor{For}{for}{\string:}{}%
\SetKwIF{If}{ElseIf}{Else}{if}{:}{elif}{else:}{}%
\SetKwFor{While}{while}{:}{fintq}%
% \renewcommand{\forcond}{$i$ \KwTo\Range{$n$}}
\AlgoDontDisplayBlockMarkers\SetAlgoNoEnd\SetAlgoNoLine%


%% enumitem 
% \labelindent is defined in both IEEEtrans and
% enumitem. \let\labelindent\relax kind-of disables \labelindent
% defined in IEEEtrans, hence avoiding the name clash.
\let\labelindent\relax
\usepackage[inline]{enumitem}

%% algorithm2e
% \usepackage[plain]{algorithm2e}
% remove line number for one line
% \let\oldnl\nl
% \newcommand{\nonl}{\renewcommand{\nl}{\let\nl\oldnl}}

%% subfigure
\usepackage[caption=false,font=footnotesize]{subfig}
% make references to subfigures appear as \thefigure(\thesubfigure)
\captionsetup[subfigure]{subrefformat=simple,labelformat=simple,listofformat=subsimple}
\renewcommand\thesubfigure{(\alph{subfigure})}

% variables
\newcommand{\mc}[2][]{{\mathcal{#2}_{\textrm{#1}}}}
\newcommand{\q}[1]{\bm{q}_{\textrm{#1}}}
\newcommand{\qd}[1]{\bm{\dot{q}}_{\textrm{#1}}}
\newcommand{\T}[1]{\bm{T}_{\textrm{#1}}}
\newcommand{\x}[1]{\bm{x}_{\textrm{#1}}}
\newcommand{\qvect}{\bm{q}}
\newcommand{\Tvect}{\bm{T}}
\newcommand{\GP}{\mathcal{G}\cap\mathcal{P}}
\newcommand{\cP}{\mathcal{P}}
\newcommand{\cC}{\mathcal{C}}
\newcommand{\cO}{\mathcal{O}}
\newcommand{\bfp}{\mathbf{p}}
\newcommand{\calT}{\mathcal{T}}
% acronyms
\newcommand{\ie}{{\textit{i.e.}}}
\newcommand{\etal}{\textit{et~al.}}

% theorem environment
\newtheorem{theorem}{Theorem}
\newtheorem{proof}{Proof}
\newtheorem{lemma}{Lemma}
\newtheorem{proposition}{Proposition}
\newtheorem{corollary}{Corollary}
\newtheorem{remark}{Remark}
% TODO
\newcommand{\TODO}[1]{\noindent {\color{red} \{{\bf To-do:} #1\}}}
% COMMENT
\newcommand{\comment}[1]{}

% change tt font
\renewcommand{\tt}{\fontfamily{cmtt}\selectfont}

% figures path
\graphicspath{{figures/}}

% \overrideIEEEmargins
% set margins
\setlength{\floatsep}{2pt plus 1pt minus 1pt}
\setlength{\textfloatsep}{5pt plus 1pt minus 2pt}

%% TITLE
\title{MAS714-Homework 2}
%% AUTHOR
\author{Pham Tien Hung, Zhang Xu
}

%% DATE
\date{}


%%%%%%%%%%%%%%%%%%%%%%%%%%%%%%%%%%%%%%%%%%%%%%%%%%%%%%%%%%%%%%%%%%%%%%
\begin{document}
\maketitle
\section*{Excercise 1}
\subsection*{a)}
Prove that every tree is a bipartite graph.
\begin{proof}
	We will prove by induction, that is very tree $T(V, E)$ whose
	$|V| = n$ for all $n$ is a bipartite graph. The case $n=2$ is trivially true.

	Assume that this is true for $|V| = n$, we will now prove that it
	is true for every tree $T'$ whose $|V'| = n + 1$. 

	Consider one such tree $T'$ with $|V'| = n + 1$, we select a leaf node $u$
	and remove it from $T'$ creating $T''$. Clearly, $|V''| = n$ and
	therefore $T''$ is now a bipartite graph. Let $L''$ and $R''$ be the
	corresponding bipartite set of the tree $T''$.

	Without loss of generality, assume that the node $u$ connects to
	some node $v$ belonging to $L''$, we construct two new set $L'$
	and $R'$:
	\[
		L' = L''
	\]
	\[
		R' = R'' \cup \{u\}
	\]
	Clearly, these two sets imply that $T'$ is bipartite.	
\end{proof}

\subsection*{b)}
\begin{algorithm}[h]
  \caption{Check for bipartite graph ($G(V, E)$)}
    T = Kruskal($G$) \;
    L, R are bipartite partition of vertices from $T$ \;
    \For{$(u, v)$ in $E$}{
    	\If{both $u, v$ belongs to L or R}{
    		\Return False
    	}
    }
    \Return True
\end{algorithm}

Time complexity: \TODO{}

\begin{proof}
	\TODO{}
\end{proof}

\end{document}
